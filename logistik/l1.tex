\section{Einführung in die Logistik und SCM}

\textbf{Logistik}:
\begin{itemize}
	\item \textbf{Definition}: 
	\begin{itemize}
		\item Planung, Implementierung und Kontrolle 
		\item von effizienten, effektiven Vor- und Rückflüssen 
		\item sowie der Lagerung von Gütern, Dienstleistungen und Informationen 
		\item zwischen Ursprungs- und Verbrauchsort 
		\item mit dem Ziel, die Kundenanforderungen zu erfüllen
	\end{itemize}
	\item \textbf{Aufgabe der Logistik} ist es,
	\begin{itemize}
		\item den Kunden mit dem richtigen Produkt, am richtigen Ort, zur richtigen Zeit,
		\item unter gleichzeitiger Optimierung eines vorgegebenen Leistungskriteriums (z. B.
		Minimierung der Gesamtkosten),
		\item und unter Berücksichtigung gegebener Anforderungen (z. B. Servicegrad) und
		Beschränkungen (z. B. Budget) zu versorgen
	\end{itemize}
	\item \textbf{7 R's der Logistik}:
	\begin{itemize}
		\item Richtiges Produkt
		\item Richtige Zeit
		\item Richtiger Ort
		\item Richtige Menge
		\item Richtige Qualität
		\item Richtige Kosten
		\item Richtige Information
	\end{itemize}
	\item \textbf{Auf was bezieht sich Logistik heute?}
	\begin{itemize}
		\item Alle arbeitsteiligen Wirtschaftssysteme, in denen es auf zeit-, kosten- und
		mengenabhängige Verteilung von Gütern und Dienstleistungen ankommt
	\end{itemize}
\end{itemize}
\bigskip
\textbf{Supply Chain}: 

