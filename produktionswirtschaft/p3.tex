\section{End-of-Life in der Circular Economy}

\textbf{Reverse Supply Chain Planning:} Alle Aktivitäten, die notwendig sind, um
\textbf{End-of-Life} (EoL) Produkte zu sammeln und zu bearbeiten 

$\rightarrow$ Restwert erhöhen oder negative Auswirkungen minimieren oder so bearbeiten, dass sie wieder in Wertschöpfungsprozesse einfließen können

\textbf{Einteilung von Mehrwegsystemen:}
\begin{itemize}
	\item \textbf{Betriebseigene Mehrwegsysteme (Insellösungen)}:
	\begin{itemize}
		\item Anschaffung von individuellen Mehrwegverpackungen
		\item \textbf{Vorteile}: Branding der Verpackungen, bessere Anpassbarkeit an Produkte
		\item \textbf{Nachteile}: Hohe Kosten durch geringe Anzahl benötigter Verpackungen, Unternehmerisches Risiko
	\end{itemize}
	\item \textbf{Gemeinsame Mehrwegsysteme (Poollösungen)}:
	\begin{itemize}
		\item Zusammenschluss mehrere Unternehmen, Beziehung der gleichen Verpackungen von einem Poolanbieter
		\item \textbf{Vorteile}: Geringere Kosten, Schwankungen der Nachfrage können ausgeglichen werden, Flexible Rückgabe für den Kunden
		\item \textbf{Nachteile}: Standards der Mehrwegbehälter müssen eingehalten werden, ggf. müssen Kosten getragen werden, die nicht vom eigenen Kunden verursacht wurden
	\end{itemize}
\end{itemize}

\textit{Rechenbeispiel s. Produktion Übung, F19-20}

\pagebreak
\textbf{Kosten eines Mehrwegsystems:}
\begin{itemize}
	\item \textbf{Ökonomische Kostenkomponenten}: Einkaufspreis, Lagerungskosten, Transportkosten
	\item \textbf{Ökologische Kosten}: Transportwege, Verluste, Verpackungsmaterialien
\end{itemize}